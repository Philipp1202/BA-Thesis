% !TEX root = ../Thesis.tex
\chapter{Abstract}
Text-entry is one of the most common forms of computer-human interaction and indispensable for many tasks such as word processing and some approaches to multimedia retrieval. The conventional keyboards everybody knows have long-established as the main text input method for desktop and laptop computers and even for touchscreen based devices they are very useful. But when it comes to virtual reality (VR) and augmented reality (AR), with today's technology, it lacks of tactile feedback and accurate finger tracking. As a result, text input for VR and AR is still an area of active research.\\
In recent years, word-gesture typing/ slide-to-type keyboards have been introduced in most major smartphone operating systems. To show the power of such keyboards, we can make a comparison between the best possible performance on a conventional keyboard with the qwerty layout and a word-gesture keyboard with the ATOMIK layout. MacKenzie and Zhang \cite{10.1145/302979.302983} found, that after about 17 hours of practicing, the user of a conventional keyboard with qwerty layout could input about 45 words per minute. On the other hand, Zhai and Kristensson \cite{Kristensson2004SHARK2AL} had in their experiment with a word-gesture keyboard with the ATOMIK layout a record input speed of about 52 to 86 words per minute. This shows, that the potential of such word-gesture keyboards is high and one could really write faster after some training. Therefore, we may ask ourselves if this could also be an efficient text input method for VR/AR.