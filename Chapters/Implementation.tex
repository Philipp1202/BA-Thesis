% !TEX root = ../Thesis.tex
\chapter{Implementation}

In this section, we will introduce how we implemented a word-gesture keyboard using Unity and a python script and how we used SHARK2 for our algorithm.

\section*{Word Graph Generator}
As mentioned in the related-work SHARK2 part TODO: REFERENCE TO MENTIONED SECTION , to make SHARK2 work, the perfect graphs for all the words in a lexicon are needed. Therefore, we needed a script, that produces or overwrites a file for every available keyboard layout and writes the sampled points of every graph in it. Such a file contains per line a word, then a certain number of points from the word's perfect graph followed by the same points, but normalized (normalized as mentioned in the related-work SHARK2 section) TODO: REFERENCE TO MENTIONED SECTION.\\
To run the script, the user has to provide the name of the layout that he/she wants to create the perfect graphs for. Additionally, he/she has also to write the name of the text file containing all the words (lexicon). The script then either creates a new file named "sokgraph\_\textit{layout}.txt" or if already a file with this name exists, it deletes its content to write new in it. Then it fills the file line by line as mentioned above.\\
The file can only be executed for one layout a time. Hence, if there are more available layouts for our word-gesture keyboard, the user has to run the script several times.

\section{Used Algorithm}
For our word-gesture keyboard we used a weaker version of SHARK2. This means, we do also work with two channels, a location channel and a shape channel. The shape channel is to calculate the deviation from the user inputted graph and a perfect graph from a word in the sense of distance with respect to their shape, the location channel is for the same thing, but not for the shape, but rather the position. When looking at the shape, we have to normalize the graphs in a specific way, so the position, where they exactly lie in a coordinate system does not matter. When looking at the position, we look at the graphs as they are, without normalizing or changing anything. As in the SHARK2 system we also use the start and end positions of the graphs as pruning method. The difference is, that for SHARK2, the authors chose to use normalize all the graphs in scale and translation before comparing. We do not normalize the graphs, but just look at the start and end positions of a user input graph and a word's graph. Another thing we have almost implemented the same as it is in SHARK2 is $\delta$. For the shape and location channel integration, the used $\delta$ in SHARK2 is equal to the radius of one key. We do the same for the location channel (in the integration part), but we do not use the same $\delta$ for the shape channel (in the integration part). For this, we take a $\delta$ that equals the radius of a normalized key. That means, a small graph will have a bigger delta than a big graph, because we do normalize the graphs and a small graph gets stretched by it, whereby a big graph gets drawn together. TODO: REVISE.\\
However, we do currently not use any language information nor dynamic channel weighting by gesturing speed.

\section{Functions}
The most important, but also most basic function our word-gesture keyboard provides, is the writing of words with gestures. A user can press the trigger button of a VR controller inside the keyboards hitbox and start making a gesture on it. The user will see a TODO: SEE COLOR. red/purple line that is drawn on the keyboard where he/she moves. This helps the user to keep track of the trace he/she drew. When the user wants to finish the gesture, he/she needs to release the trigger of the controller. At this moment, our program starts to evaluate the 5 words with the highest accordance to the user inputted graph. The one with the highest accordance will be written into the text field. The other 4 are displayed at the keyboard TODO: CITE TO PICTURE WITH RECOMMENDATIONS., such that the user also can choose between these. When the user chooses one of these 4 words, the word that has been written into the text field before is getting replaced by the user's chosen word.

"draw" words with gestures/put single letters
create own layout
change size
---choose between best 5 words
keyboard is movable
chose between available layouts all the time possible
add new words
