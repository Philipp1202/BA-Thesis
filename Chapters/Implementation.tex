% !TEX root = ../Thesis.tex
\chapter{Implementation}

In this section, we will introduce how we implemented a word-gesture keyboard using Unity, a python script and the $\text{SHARK}^2$ system.

\section{Word Graph Generator}
As mentioned in the background section, in the part about $\text{SHARK}^2$ (\ref{SHARK2}), to make $\text{SHARK}^2$ work, the perfect graphs for all the words in a lexicon are needed to be precalculated. Therefore, we needed a script, that either creates or overwrites a file for every available keyboard layout. It has to write the words included in the lexicon together with the corresponding $N$ sampled points of every graph in it. Such a file contains per line a word, then a certain number of points from the word's perfect graph followed by the same points, but normalized (normalized as mentioned in the background section about $\text{SHARK}^2$ (\ref{normalize}))\\
To run the script, the user has to provide the name of the layout that they want to create the perfect graphs for. Additionally, they also have to write the name of the text file containing all the words (lexicon). The script then either creates a new file named "sokgraph\_\textit{layout}.txt" or if already a file with this name exists, it deletes its content. Then it fills the file line by line as mentioned above.\\
The file can only be executed for one layout at a time. Hence, if there are more available layouts for our word-gesture keyboard, the user has to run the script for every single one, which can take a bit of time. To calculate and write a file for all about 10'000 words in our used lexicon, it takes about four to five seconds.\\
One thing the script pays attention to is, if a word can be written with used layout. If there are words in the lexicon, that can not be written with the given layout, the script skips this one. Hence, there will be no line in the file for said word. 

\section{Used Algorithm}
For our word-gesture keyboard we used a weaker version of the algorithm used in $\text{SHARK}^2$. This means, we do also work with two core channels, a location recognizer and a shape recognizer. The shape recognizer is to calculate the deviation from the user inputted graph and a perfect graph from a word in the sense of distance with respect to their shape. The location recognizer is for the same thing, but not with respect to the shape, but rather the positioning. When looking at the shape, we have to normalize the graphs in a specific way (as mentioned in section (\ref{normalize})), so the position, where they exactly lie in a coordinate system does not matter. When looking at the position, we look at the graphs as they are, without normalizing or changing anything.\\
As in the $\text{SHARK}^2$ system we also use the start and end positions of the graphs as pruning method. The difference is, that for $\text{SHARK}^2$, the authors chose to normalize all the graphs in scale and translation before comparing the start-to-start and end-to-end positions. We do not normalize the graphs, but just look at the start and end positions of a user input graph and a word's graph. As threshold we set the width of a key. That means, if either the start-to-start or end-to-end position is bigger than a key width, the word to which this applies, gets discarded.\\
Another thing we implemented different is $\sigma$. For the channel integration formula \ref{eqn:gaussian}, they used $\sigma$ in $\text{SHARK}^2$ as a parameter. They determine its value by the gesturing speed (\ref{gesturing speed}). We do not use the gesturing speed in our algorithm. For the location channel (in the formula \ref{eqn:gaussian}) we use the radius of a key TODO: IS IT WIDTH OR RADIUS? as value for $\sigma$. For the shape channel (in the formula \ref{eqn:gaussian}) we use a variable value. We take a value that equals the radius of a normalized key. That means, a small graph will have a bigger $\sigma$ than a big graph, because we do normalize the graphs longer bounding box side to a fixed length. Therefore, a small graph gets stretched by it, whereby a big graph gets contracted.\\
But as mentioned in the beginning of this section, we use a weaker form of the algorithm used in $\text{SHARK}^2$. That being said, we do currently not use any language information nor dynamic channel weighting by gesturing speed.

\section{Functions}
\subsection{Text input}
There should be two text input methods for the user. One is the input of words with gestures, the other one is the input of single characters by tapping single keys.
\subsubsection{With Gestures}
The most important, but also most basic function our word-gesture keyboard provides, is the writing of words with gestures. A user can press and hold the trigger button of a VR controller inside the keyboard's hitbox and start making a gesture on it. The user will see a TODO: SEE COLOR. red/purple line, that is drawn on the keyboard where they move. This helps them to keep track of the trace they drew. When the user wants to finish the gesture, they need to release the trigger of the controller. At this moment, our program starts to evaluate the 5 words with the best match to the user inputted graph. The one with the highest confidence score will be written into the text field. The other 4 are displayed at the keyboard TODO: CITE TO PICTURE WITH RECOMMENDATIONS., such that the user can also choose between them. When the user chooses one of these 4 words, the word that has been written into the text field before, is getting replaced by the user's chosen word and the key, where the chosen word was written, will then display the replaced word.\\
\subsubsection{As Single Characters}
If the wanted word is not in the lexicon, there will not be any entry in the "sokgraph\_\textit{layout}.txt". Therefore, our algorithm will not be able to get this wanted word as best match, hence it can not be written as gesture. For this case, we need a method to input single characters. Fortunately with our word-gesture keyboard this almost works without additional work. If the single letters are as graphs in the layout files, TODO: DEFINE LAYOUT FILES AS SOKRAPH\_LAYOUT.TXT SOMEWHERE. the user is more or less able to write single letter with a gesture. This "gesture" would just be a point on the right key, hence a single click at the right position. But there might be a little inconvenience. This is caused by the fact, that we work with distances. The distance from one letter to another is not too big. And if for example the user wants to write an "e" but presses the key with the "e" on it on its left side and not perfectly in the middle, our system would also evaluate that aside from "e", also "w" and "we" are words, the user might have intended to write (on a conventional qwertz or qwerty layout). To get rid of this, our system checks, if the user inputted graph's bounding box is smaller than TODO: HOW MUCH SMALLER?. It recognizes, that the user wants to write a single letter, and then takes the best match. To get back to the example, "we" would be discarded and "e" would get a higher score than "w", because of a smaller location channel distance. Therefore, the written "word" in this case would be "e".\\

\subsection{Create New Layouts}
Another function is the creation of own layouts. A layout text file exists, that contains all available layouts. The user can create as many new layouts as they want to. To create a new one, the user has to write the new layout's name on the first line and on the following lines write the characters in an order, in which they want to have them on the keyboard. All the unicode characters should be working, but two. In the current implementation, one whitespace is used to declare the position of the spacebar and the "<" character is used for the backspace key. The layout text file gets read at the start of the program, so it cannot be edited while the program is running, or to be precise, the changes will not be recognized during runtime. One smaller thing we implemented is, that at the start of the program all characters used in the layout not yet in the lexicon text file and "sokgraph\_\textit{layout}.txt" are being added. Without doing this, the user would not be able to input any single special character with their newly created keyboard, because the system simply would not find it in the "sokgraph\_\textit{layout}.txt" file. During the runtime, the user then is able to switch between available layouts.

\subsection{Options}
We implemented an options button for out keyboard. It is marked with a black gear. The user can click on it with the trigger button of the controller, when being in the hitbox of it. Then, three new buttons appear. These are the currently available options. One is to add new words, one to change the currently selected layout and the last one to scale the keyboard.

\subsubsection{Add Words}
Because the whole system works with a lexicon full of words and only these can be written as gestures, there will also be words, the user wants to write, that are not yet in said lexicon. For this case, we implemented a function such that the user can add new words. They can access it via the options button. Then they have to press the button where it says "add word". When this button is pressed another button appears, the "add word to dict" button TODO: REVISE. When the user presses this one, the word (it does not need to be one, that really exists and can not include a space) displayed above the keyboard, will be added to the lexicon text file. Additionally, for every available layout, the corresponding graph for the newly added word will be added to the corresponding "sokgraph\_\textit{layout}.txt" text file. One additional thing we implemented is, that the word's graph only gets appended in the text file, if this word can be written with the layout the file corresponds to. For example, if a user wants to add the word "öffentlich", but they made a layout without the letter "ö", this word could never be written with said layout, hence it would be unnecessary to have this word in said "sokgraph\_textit{layout}.txt" text file.

\subsubsection{Change Scale and Layout}
There are other two functions that can be found under the options. One is to change the size of the keyboard. When the user clicks on the "change layout" button, a "+" and a "-" button appear. By pressing the "+" button, the keyboard gets bigger, by pressing the "-" button, the keyboard gets smaller. This can help the user to use the keyboard. Depending on how they want to move their arm, they can make it slightly bigger or smaller and put the keyboard a bit closer or further away.\\
The other function is the ability to change the layout. The user gets a list of all available layouts, when pressing the "change layout" button. They then can click on one of these layouts and the keyboard will change its appearance.

\subsection{Other}
To end the implementation part, we will shortly talk about the last function, the ability to grab andd move the keyboard. This means, the user can grab the keyboard, move it around in the room and rotate it as they want. We were able to do this thanks to script that already existed in vitrivr TODO: HOW TO REFERENCE THIS? and did not have to implement anything on our own.



---"draw" words with gestures/put single letters
---choose between best 5 words
---create own layout
---add new words
---chose between available layouts all the time possible
---change size
---keyboard is movable
