% !TEX root = ../Thesis.tex
\chapter{Implementation}

In this section, we will introduce how we implemented a word-gesture keyboard using Unity and a python script and how we used SHARK2 for our algorithm.

\section*{Word Graph Generator}
As mentioned in the related-work SHARK2 part TODO: REFERENCE TO MENTIONED SECTION , to make SHARK2 work, the perfect graphs for all the words in a lexicon are needed. Therefore, we needed a script, that produces or overwrites a file for every available keyboard layout and writes the sampled points of every graph in it. Such a file contains per line a word, then a certain number of points from the word's perfect graph followed by the same points, but normalized (normalized as mentioned in the related-work SHARK2 section) TODO: REFERENCE TO MENTIONED SECTION.\\
To run the script, the user has to provide the name of the layout that he/she wants to create the perfect graphs for. Additionally, he/she has also to write the name of the text file containing all the words (lexicon). The script then either creates a new file named "sokgraph\_\textit{layout}.txt" or if already a file with this name exists, it deletes its content to write new in it. Then it fills the file line by line as mentioned above.\\
The file can only be executed for one layout a time. Hence, if there are more available layouts for our word-gesture keyboard, the user has to run the script several times.

\section{Used Algorithm}

\section{Functions}

"draw" words with gestures
create own layout
change size
choose between best 5 words
keyboard is movable
chose between available layouts all the time possible
add new words