% !TEX root = ../Thesis.tex
\chapter{Introduction}

\section{Motivation}
Lots of virtual reality applications need some kind of text input method. While most of the time this function is provided by a conventional keyboard implemented for VR, this might not be the handiest one. To use this kind of keyboard, a user has to tap on single letter key to input single characters. Even though most of the text inputs in VR applications may not be very long, it can still be exhausting for our arms to input these. The conventional keyboard in VR applications often has a bigger scale than a normal keyboard in reality. That said, we have to move quite a distance with our arms and always move up and down to not accidentally hit a key.\\
Now, we want to develop and provide a keyboard, that is handy to use in VR. The idea, as stated in the abstract, is to develop a word-gesture keyboard. With a word-gesture keyboard, the text input could become more comfortable. We would not need to move our arms up and down, we could just move on a flat plane from one key to another. Such a keyboard could also be smaller, because the precision is not as important as it is for the normal keyboard. For example, on a conventional keyboard, if we would tap in the middle of two keys, we cannot really tell which one to take for the input. But with a word-gesture keyboard, where we work with distances and graphs (more on this later), it has not that much of an impact. Therefore, a smaller keyboard is possible, and we do not have to move our arms that much. Hence, it might be less exhausting to write with a word-gesture keyboard.\\
To show the power of such keyboards in a non-VR environment, we can make a comparison between the best possible performance on a conventional keyboard with the qwerty layout and a word-gesture keyboard with the ATOMIK layout. MacKenzie and Zhang \cite{10.1145/302979.302983} found, that after about 17 hours of practicing, the user of a conventional keyboard with qwerty layout could input about 45 words per minute. On the other hand, Zhai and Kristensson \cite{Kristensson2004SHARK2AL} had in their experiment with a word-gesture keyboard with the ATOMIK layout a record input speed of about 52 to 86 words per minute. This shows, that the potential of such word-gesture keyboards is high and one could really write faster after some training. Therefore, we may ask ourselves if this could also be an efficient text input method for VR/AR.

\section{Goals}
For this thesis, we have two main goals. The first one is to develop a word-gesture keyboard. It has to work with vitrivr-VR and has to be available as open-source. It should also be available as a Unity package, so other developers can use it in their Unity projects as well. \\
The second goal is to evaluate said keyboard. The evaluation will be conducted according to current research standards with the usage of the MacKenzie phrase set.
