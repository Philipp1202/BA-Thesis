% !TEX root = ../Thesis.tex
\chapter{Introduction}

\section{Motivation}
To work with vitrivr-VR we need a possibility to input some text into the query text fields. While there already exists a keyboard, it is maybe not the handiest one. The currently implemented one is a conventional keyboard, where a user has to tap on single letters to input single characters. Even though these "query texts" will only consist of a single word or some few words, it still might be exhausting for our arms to input these. The keyboard in vitrivr-VR has a bigger scale than a normal keyboard in reality. We have to move quite a distance with our arms and always move up and down to not accidentally hit a key. With a word-gesture keyboard, the text input could become more comfortable. We would not need to move our arms up and down, we could just move on a flat plane from one key to another. Such a keyboard could also be smaller, because the precision is not as important as it is for the normal keyboard. For example, on a conventional keyboard, if we would tap in the middle of two keys, we cannot really tell which one to take for the input. But with a word-gesture keyboard, where we work with distances and graphs (more on this later), it has not that much of an impact. Therefore, a smaller keyboard is possible, and we do not have to move our arms that much. Hence, it might be less exhausting to write with a word-gesture keyboard.\\
Other VR applications often do also only use a normal keyboard as text input method. The word-gesture keyboard developed in this thesis will be available as open-source unity package and available for everybody. Therefore, developers of other VR applications may also be using our word-gesture keyboard if they are interested.

\section{Goals}
For this thesis, we have two main goals. The first one is to develop a word-gesture keyboard. This is a keyboard, that more or less might look like a normal one. But instead of tapping on the different keys, words are written with gestures. It has to work with vitrivr-VR and has to be available as open-source. It should also be available as a Unity package, so other developers can use it in their Unity projects as well. \\
The second goal is to evaluate said keyboard. The evaluation will be conducted according to current research standards with the usage of the MacKenzie phrase set.
