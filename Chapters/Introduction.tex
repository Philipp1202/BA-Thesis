% !TEX root = ../Thesis.tex
\chapter{Introduction}

\section{Motivation}
Lots of virtual reality applications need some kind of text input method. While most of the time this function is provided by a non-physical conventional keyboard implemented for VR, this might not be the most convenient solution. To use this kind of keyboard, a user has to tap on single letter keys to input single characters. Even though most of the text inputs in VR applications may not be very long, it can still be exhausting for the arms to input these. The conventional keyboard in VR applications often has a bigger scale than a physical one in reality. That said, a user has to move quite a distance with their arms and always move up and down to not accidentally hit a wrong key. Now, there is the idea to develop another type of keyboard, a so-called word-gesture keyboard, which is possibly more practical to use in VR. Such a keyboard looks more or less like a conventional one but instead of tapping on the different keys, words are written with gestures. With a word-gesture keyboard, the text input could become more comfortable. A user would not need to move their arms up and down, they could just move on a flat plane from one key to another. Such a keyboard could also be smaller because the precision is not as important as it is for the conventional keyboard. For example, if we tap in the middle of two keys on a conventional keyboard, we cannot tell which one to take for the input. But with a word-gesture keyboard, where we work with distances and graphs (more on this later), it does not have that much of an impact. Therefore, a smaller keyboard is possible, and we do not have to move our arms that much. Hence, it might be less exhausting to write with a word-gesture keyboard.\\
To show the power of such keyboards in a non-VR environment, we can make a comparison between the best possible performance on a conventional keyboard with the QWERTY layout and a word-gesture keyboard with the ATOMIK layout. MacKenzie and Zhang \cite{MacKenzie_Zhang_SoftKeyboard} found that after about 17 hours of practicing, the user of a conventional keyboard with QWERTY layout could input about 45 words per minute. On the other hand, Zhai and Kristensson \cite{Kristensson2004SHARK2AL} measured in their experiment with a word-gesture keyboard with the ATOMIK layout a record input speed of about 52 to 86 words per minute. This shows that the potential of such word-gesture keyboards is high and one could write really fast after some training. Therefore, we may ask ourselves if this could also be an efficient text input method for VR and AR.

\section{Goals}
For this thesis, we have two main goals. The first one is to develop a word-gesture keyboard. It has to work with vitrivr-VR and has to be available as open-source. It should also be available as a Unity package so that other developers can use it in their Unity projects as well.\\
The second goal is to evaluate said keyboard. The evaluation will be conducted according to current research standards with the usage of the MacKenzie phrase set. With the evaluation, we want to find out whether a word-gesture keyboard is a viable text input method for VR. To anaylze this, we need to take a look at the writing speed and different error rates.