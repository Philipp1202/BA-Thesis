% !TEX root = ../Thesis.tex
\chapter{Evaluation}

In this section we want to talk about the evaluation as a whole. We want to look at the phrases we took, how we carried it out, the results observed and shortly discuss what all of this means.

\section{MacKenzie Phrase Set}
One precondition for the evaluation was to use the MacKenzie Phrase Set\footnote{http://www.yorku.ca/mack/PhraseSets.zip}. Basically, this is just a set of 500 phrases. According to the paper \cite{10.1145/765891.765971}, such a phrase set should use phrases of moderate length, that are easy to remember and representative for the target language. These phrases do not contain any punctuation. Some of them use uppercase characters, but the authors mention, that participants can also be instructed to ignore the case of the characters. 
%The complete MacKenzie Phrase can be downloaded at \hyperlink{http://www.yorku.ca/mack/PhraseSets.zip}{http://www.yorku.ca/mack/PhraseSets.zip}.
\\
Some statistics for the whole phrase set, also found in the original paper \cite{10.1145/765891.765971}: The MacKenzie phrase set consists of 500 phrases, that have a minimum length of 16, a maximum length of 43 and an average length of 28.61 characters. On the whole, 2712 words were used, which consist of 1163 unique words. A phrase consists of a minimum of 1, a maximum of 13 and on average of 4.46 words.

\section{Task of the Participants}
The task of the participants was to copy 15 "random" phrases. They are not really random, but adjacent phrases from the downloadable MacKenzie Phrase Set (\url{http://www.yorku.ca/mack/PhraseSets.zip}{http://www.yorku.ca/mack/PhraseSets.zip}). As they are not in a specific order, e.g. alphabetic order, we decided to do it like this.\\
TODO: PICTURE OF EVALUATION SCENE.
The participants could see two text fields. On the top was the phrase to copy, on the bottom the words/phrase they wrote. If the given phrase matched the user inputted phrase, a sound would sound, such that the participants knew when they finished one specific phrase. After that, a new phrase would appear until 15 phrases were correctly inputted. If an incorrect word was entered, the user either could use the word suggestions (fig \ref{fig:write_suggestions}) or delete the wrong word and try to write it again. If a mistake was only noticed later on, the participants had to remove all words and characters up to and including the wrong word by using the backspace button.\\
After this first step, in the second step, we shortly explained two functions of the keyboard, which they could test afterwards. First the scaling buttons and then the function to add a new word. This is important, because we wanted to know if they found these functions useful and well implemented.\\
The last step of the evaluation was to fill a questionnaire. First it had some general questions about the participant's experience in VR. Then there was a block of questions in the form of a system usability scale. Per question, there were five possibilities to set the cross. From 1 (strongly disagree) to 5 (strongly agree). The questions are structured in such a way, that if the user was highly satisfied with everything, they would alternately make a cross at the 1 and 5. TODO: FRAGEBOGEN ALS ANHANG BEIFÜGEN.

\section{Carry-out}
To carry out the evaluation, we used two different VR systems. One was a setup with a HTC vive and HTC vive controllers. The other one included an Oculus Rift headset with corresponding controllers. TODO: GET RIGHT NAME AND STUFF OF VR STUFF. Even though these are two different systems, it does not change much for the participants. In fact, only the controllers differ a bit.\\
To find participants, we wrote an email to students from our university, and asked family members and friends. All in all, eleven people got in touch with us and participated at our evaluation. Every participant got the same explanation to give everybody the same foundation of knowledge.\\
We told them that the keyboard is movable, if they are close enough to the keyboard (the color then gets a bit brighter) and press and hold the controller's grip button. If they release it, the keyboard gets static again and stays where it got put. To write, they do also have to be in the hitbox of the keyboard but not pressing and holding the grip button, but the trigger button. Then they had to make a gesture over the characters of the keyboard to write a word. We also told them, that if they do a full gesture and a word longer than one character is written, a space is automatically put behind the word. We also said to them, that single characters could be inputted by clicking on a key of the intended character. If they did so, no space is put, and they have to do it their own. In the English language, this is particularly important for the words "I" and "a". We also told them, that if they made a gesture and a word was written, there may be one to four other choosable words they could pick from, if the word written in the text field is the wrong one. They were also informed about the backspace. So, that if they use the backspace button after writing a word, the whole word gets deleted and afterwards only single characters get deleted. We also told the participants, that we have enough time and that they should not hurry, but rather look, that the inputted words are correct. Because if they are not correct, they have to use the backspace a lot of times. We especially mentioned the word ``the''. All the time ``thee'' would be written as the best match, therefore they would have to correct it every time.\\

\section{Results}



\section{Discussion}