% !TEX root = ../Thesis.tex
\chapter{Evaluation}

In this section we want to talk about the evaluation as a whole. We want to look at the phrases we took, how we conducted it, the results observed and shortly discuss what all of this means.

\section{MacKenzie Phrase Set}
One precondition for the evaluation was to use the MacKenzie Phrase Set TODO: CITE TO PAPER. Basically, this is just a set of 500 phrases. According to the paper, such a phrase set should use phrases of moderate length, that are easy to remember and representative for the target language. These phrases do not contain any punctuation. Some of them use uppercase characters, but the authors mention, that participants can also be instructed to ignore the case of the characters. The complete MacKenzie Phrase can be downloaded at \hyperlink{http://www.yorku.ca/mack/PhraseSets.zip}{http://www.yorku.ca/mack/PhraseSets.zip}.\\
Some statistics for the whole phrase set, also found in the original paper TODO: CITE PAPER: The MacKenzie phrase set consists of 500 phrases, that have a minimum length of 16, a maximum length of 43 and average length of 28.61 characters. On the whole, 2712 words were used, which consist of 1163 unique words. A phrase consists of a minimum of 1, a maximum of 13 and on average of 4.46 words.

\section{Task of the Participants}
The task of the participants was to copy 15 "random" phrases from the MacKenzie phrase set. They are not really random, but adjacent phrases in the downloadable MacKenzie Phrase Set (\hyperlink{http://www.yorku.ca/mack/PhraseSets.zip}{http://www.yorku.ca/mack/PhraseSets.zip}). As they are not in a specific, e.g. alphabetic order, we decided to do it like this.\\
The participants could see two text fields. On the top was the phrase to copy, on the bottom the words/phrase they wrote. If the given phrase matched the user inputted phrase, a sound would sound, such that the participants knew when they finished one specific phrase. After that, a new phrase would appear until 15 phrases were correctly inputted.\\
After this first step, in the second step, we shortly explained two other functions of the keyboard. First the scaling buttons and then the function to add a new word. This is important, because we wanted to know if they found these functions useful and well implemented.\\
The last step of the evaluation was to fill a questionnaire TODO: ALS ANHANG BEIFÜGEN.

\section{Carry-out}
The VR system we used are the HTC vive TODO: GET RIGHT NAME AND STUFF OF VR STUFF.
To find participants, we wrote an email to students from our university, and asked family members and friends not involved in the process of making this bachelor's thesis. All in all, TODO: HOW MANY PEOPLES? people got in touch with us and participated at our evaluation. Every participant got the same explanation. We told them that the keyboard is movable, if they are close enough to the keyboard (the color then gets a bit brighter) and press the controller's grip button. To write, they do also have to be in the hitbox of the keyboard but not pressing the grip button, but the trigger button. We also told them, that if they do a full gesture and a word longer than one character is written, a space is automatically put behind the word. If they put in a single character, by clicking on that key of this character, no space is put, and they have to do it their own. This is particularly important for the words "I" and "a". We told also told them, that if they made a gesture and a word was written, there may be one to four other choosable words they could pick, if the word displayed as first is the wrong one. We also told the participants, that we have enough time and that they should not hurry, but rather look, that the inputted words are correct. Because if they are not correct, they have to use the backspace a lot of times.\\

\section{Results}



\section{Discussion}