% !TEX root = ../Thesis.tex
\chapter{Related Work/ Background}

In this chapter we introduce the environment the word-gesture keyboard is mainly developed for, some things about the normal, classical keyboard and word-gesture keyboard in general, with an excursion about SHARK2.

\section{vitrivr-VR and UnityVR}
Vitrivr\footnote{https://dbis.dmi.unibas.ch/research/projects/vitrivr-project/} is an open source full stack content-based multimedia retrieval system. It supports image, audio and 3D collections and features a very broad set of query paradigms that are supported. Vitrivr was developed by the Database and Information Systems group\footnote{https://dbis.dmi.unibas.ch} (dbis) of the university of Basel. For our thesis, we use the VR part of vitrivr, namely vitrivr-VR. This is being developed in Unity\footnote{https://unity.com}. Unity is a tool for developers, where one can create projects in 2D, 3D and VR. To a certain degree, Unity is free to use, but one can provide assets that are not for free. In Unity, a developer can also provide packages. These can be imported and used by other developers in their Unity projects.

\section{Classical keyboard}
A classical keyboard is the most used keyboard type. On desktop and laptop computers we normally use such a keyboard. One thing that might be different in some countries is the layout. The layout does not change the functionality but can influence the appearance. This also applies to the most used keyboard for phones, tablets and other touchscreen-based devices. The only difference is that we do not press physical keys, but tap on the screen, where a certain key is. These keyboards work really well for text input with the previously mentioned devices. But when it comes to virtual reality (VR) or augmented reality (AR), this may not be the best possible text input method. Right now, it lacks of tactile feedback and accurate finger tracking. While this could be improved during the next years, yet it is not really there. Another reason is the size of such keyboards in VR. Due to the lack of accurate finger tracking, we have to tap on the keys with our controllers. If the keys are too close together, it might cause a problem in recognizing which key was pressed. Therefore, there needs to be either bigger keys or bigger spaces between two adjacent keys. This results in a bigger keyboard, which results in more needed movement with the arms. If we have to move our arms a lot to input some text, this can quickly become exhausting.

\section{Word-Gesture Keyboard}
A word-gesture keyboard may look pretty much the same as a classical keyboard described in the last section, but works completely different. Independent of the details of the implementation, every word-gesture keyboard (also called slide-to-type keyboard) works with gestures. That means, instead of tapping on single keys, we have to draw one line or a shape on the keyboard. This will then be evaluated by an algorithm, that determines the closest word from a lexicon. Here closeness is determined by shape comparison between the user input and a word from the lexicon. 

\subsection{SHARK2}
SHARK2 is a system introduced by Zhai and Kristensson (2003) 