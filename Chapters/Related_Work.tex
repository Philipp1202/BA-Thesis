% !TEX root = ../Thesis.tex
\chapter{Related Work/ Background}

In this chapter we introduce the environment the word-gesture keyboard is mainly developed in and developed for, some things about conventional and word-gesture keyboards in general and SHARK2.

\section{vitrivr-VR and UnityVR}
vitrivr\footnote{https://vitrivr.org/} is an "open source full stack content-based multimedia retrieval system"\footnote{https://dbis.dmi.unibas.ch/research/projects/vitrivr-project/}. It supports video, image, audio and 3D collections. It also features a very broad set of query paradigms that are supported. vitrivr was developed by the Database and Information Systems group\footnote{https://dbis.dmi.unibas.ch} (dbis) of the university of Basel. For our thesis, we use the VR part of vitrivr, namely vitrivr-VR. This is being developed in Unity\footnote{https://unity.com}.\\
Unity is a tool for developers, where one can create projects in 2D, 3D and VR. To a certain degree, Unity is free to use. Developers can provide assets and Unity packages. These can be either free to use or have to be bought. Another developer then can import and use these in their own Unity projects. The main language used in Unity is C\#. A developer can write such C\# scripts and if needed attach them to objects in a scene. These scripts can control the objects and what they are doing when a user interacts with them or something particular happens. 

\section{Conventional Keyboard}
In this thesis, when we talk about conventional keyboard, we do not look at its type of construction, if it's a mechanical keyboard or not. We also do not consider a special layout, when we talk about conventional keyboards. We define the term "conventional keyboard" as the most used keyboard type. The one were a user has to input every single letter by tapping the right key or touching the screen at the right place.\\
On desktop and laptop computers we normally use such a conventional keyboard. One thing that might be different in some countries is the layout. But that does not change the functionality. Conventional keyboards are also the most used keyboard for phones, tablets and other touchscreen-based devices. The only difference is that we do not press physical keys, but tap on the screen, where a certain key is. These keyboards work really well for text input with the previously mentioned devices. But when it comes to virtual reality (VR) or augmented reality (AR), it seems, that this is not the best method to input text. On reason for this statement is, that right now, it lacks of tactile feedback and accurate finger tracking. While this could be improved during the next years, yet it is not really there. Another reason is the size of such keyboards in VR. We have to tap on the keys with our controllers. If the keys are too close together, it might cause a problem in recognizing which key the user wanted to press. Therefore, there needs to be either bigger keys or bigger spaces between two adjacent keys. This results in a bigger keyboard, which results in more needed movement with the arms. If we have to move our arms a lot to input some text, this can quickly become exhausting.

\section{Word-Gesture Keyboard}
A word-gesture keyboard may look pretty much the same as a conventional keyboard described in the last section, but works quite different. First of all, it does not exist in a hardware version like the conventional keyboard does. It is more like the soft keyboard version, which is a keyboard displayed on a screen, like the ones used when typing text on a phone.\\
Independent of the details of the implementation, every word-gesture keyboard (also called slide-to-type keyboard) works with gestures. That means, instead of tapping on single keys, the user has to draw one line or a shape on the keyboard. This will then be evaluated by an algorithm, that determines the closest word, the one with the most similar shape seen from different aspects, from a lexicon. For example, to input the word "science", the user has to put the finger on the screen, where the "s-key" is displayed. Then they have to move, with the finger still on the screen, to the respective adjacent key with the correct character. At the end, the user has to take away their finger from the screen at the "e-key". If the gesture was more or less good, the algorithm behind should now be able to calculate, that "science" is the word, the user intended to write. But if the gesture is done bad, it can happen, that the wrong word is being calculated. 

\subsection{SHARK2}
SHARK2 is a "large vocabulary shorthand writing system for pen-based computers" \cite{Kristensson2004SHARK2AL} invented by Shumin Zhai and Per-Ola Kristensson. It can compare the user inputted graph with a perfect graph of any word in a given lexicon. A perfect graph is the graph, that is produced, if we start from the center of the word's first letter on the keyboard. Then we draw a straight line to the center of the next letter of the word and so on, until we reach the last letter.\\
The SHARK2 system needs a lexicon with words and all their perfect graphs stored. To get the most probable word from the lexicon, the user intended to write, it uses a multi-channel recognition system. Each channel alone from the system developed by Zhai and Kristensson \cite{Kristensson2004SHARK2AL} does not necessarily have enough power, but all the channels together can detect the right word. The two core channels are the shape and location recognizers.\\
First of all, $SHARK^2$ uses template pruning. It compares the start and end positions of the perfect graph of each word in the lexicon with the normalized input gesture from the user. If one of these two distances is bigger than a given threshold, the checked word will be discarded and not further considered. With normalized, the authors mean normalized in shape and location.\\
The next step is to apply the shape recognizer. It compares the shapes of the perfect graph for every word in the lexicon and the user inputted graph. For this, an amount of N sampling points has to be calculated for every graph. These N points need to be equidistant. Then they have to be normalized in scale and location. This means, that the graphs are all normalized by scaling the largest side of the graph's bounding box to a predetermined length L: 
\begin{equation}
    s = L / max(W,H)
\end{equation}
W and H are the width and height of the graph's bounding box. All points' positions have to be divided by s to get the normalized points' positions. After that, the middle point of every graph has to be set to the point (0,0). Now the distance between the user inputted graph and every word's perfect graph, that is in the lexicon, has to be calculated. To do so, we use the following formula:
\begin{equation}
    x_s = \frac{1}{N}\sum_{i = 1}^{N}\left\lVert u_i - t_i\right\rVert_2
\end{equation}
where $u_i$ is the ith point of the user input graph and $t_i$ the ith point of a word's graph. This is the so-called proportional shape matchimg distance \cite{Kristensson2004SHARK2AL}. Now, one could think, that this is enough and with the application of the template pruning and shape channel recognition, the word is perfectly determined. This is not the case. The authors stated, that words can have a similar or even same shape as other words. They call these word pairs "confusion pairs". They found out, that for example on an ATOMIK layout with a lexicon of 20'000 words, there were 1117 pairs of words that have an identical graph, if the starting and ending positions are not considered. If these are also considered with the shape, there is still a total of 537 confusion pairs.\\
To avoid these, the authors were using a second channel, not for the shape, but for the location. For the following formulas and calculations, the normalization of the graphs in not needed anymore. As the name states, it is more about the location, where the graph lies in a coordinate system, than about its shape.\\
They use an algorithm that computes the distance of the user inputted graph u to the perfect graph t of every word in the lexicon. The location channel distance is defined as:
\begin{equation}
    x_L = \sum_{i = 1}^{N}\alpha(i)\delta(i)
    \label{eqn:locationformula}
\end{equation}
where N is the number of points used to sample a graph. $\alpha(i)$ with $i \in (1,N)$ are weights for the different point-to-point distances, such that $\sum_{i = 1}^{N}\alpha(i) = 1$. $\alpha(i)$ can be valued in various ways. For $SHARK^2$ the authors used a function, that gives the lowest weight to the middle point-to-point distance. For the other point-to-point distances, the weight increases linearly towards the two ends. $\delta(i)$ is defined through following formula:
\begin{equation}
    \delta(i) =
        \begin{cases}
            0, & D(u,t) = 0 \land D(t,u) = 0 \\
            \left\lVert u_i - t_i \right\rVert_2, & \text{otherwise}
        \end{cases}
\end{equation}
where $u_i$ is the i-th point of u and $t_i$ the i-th point of t. D is defined as:
\begin{equation}
    D(p,q) = \sum_{i = 1}^{N}\text{max}(d(p_i,q) - r,0)
\end{equation}
r is the radius of a key and d is defined as:
\begin{equation}
    d(p_i,q) = \text{min}(\left\lVert p_i - q_1 \right\rVert_2, \left\lVert p_i - q_2 \right\rVert_2, \dots, \left\lVert p_i - q_N\right\rVert_2)
\end{equation}

For all these formulas N has the same definition as for formula \ref{eqn:locationformula}. The "trick" the authors use these formulas for is pretty simple. They state, that they form something like an "invisible" tunnel of one key width that contains all keys used to write a certain word. A perfect distance score of zero is given, when all the sampled points of the user inputted graph lie within the tunnel of t. If this is not the case, the distance score for t with respect to the user inputted graph is set to the sum of the spatial point-to-point distances.

With the two distances $x_S$ and $x_L$, the most probable word, the user intended to write, can be calculated. The authors assume, that the distance from a user inputted graph to the perfect graph of a word follows a Gaussian distribution. This means, if the user inputted gesture has distance x to a perfect graph of word y, the probability, that y is the intended word can be calculated using the Gaussian probability density function:
\begin{equation}
    p(x) = \frac{1}{\sigma\sqrt{2\pi}}\exp\left[{-\frac{1}{2}}\left(\frac{x-\mu}{\sigma}\right)^2\right]
\end{equation}

To do so, the authors used the following three formulas:


The final result c(w) for every word w that met the requirements tells the probability that w is the word, the user intended to write with his/her gesture.\\
The authors say, that the user can draw a graph either on visual guidance from the keyboard (looking for the next letter of a word on the keyboard) or recall from memory. A graph drawn by visual guidance results in a higher location distance score than a graph drawn from memory recall. If a user draws a graph by memory recall the location distance score will be poor and the focus lays more on the shape. Therefore, they suggest a dynamic weighting of the two channels, that is to adjust the weighting of the channels according to the time needed to draw the graph. In general, graphs drawn by memory recall are faster than visual guided ones. Hence, the gesture completion time should tell, how heavy the location channel should be weighted in the final selection. The time to complete a graph for a word obviously depends on its length and complexity. The authors then use Fitts' law (TODO: EITHER FORMULA OR REFERENCE TO IT) to calculate the normative writing time for a graph of a word. They use this result together with the actual graph production time to modify $\delta$ in (TODO: REFERENCE TO FORMULA THEY SUB DELTA).\\
The authors achieved quite satisfactory performance with the two channels, but there still might be conflicting words. To prevent these, the authors suggest to also use language information. For SHARK2 smoothed bigrams are used as the language model. It is then used to rearrange the N-best list of words received before.