% !TEX root = ../Thesis.tex
\chapter{Conclusion}

In this last section, we will briefly talk about what we have done in this thesis, the results and what can be implemented in the future to improve the keyboard developed during this thesis.

\section{Conclusion}
Overall we think, that the word-gesture approach we implemented might not be the fastest way to input words. At least this can be said for beginners. But it seems, that it has a lower error rate by far compared to single letter input methods.

\section{Future Work}
First of all, the word detecting algorithm of the keyboard can be extended with all the improvements used in the $\text{SHARK}^2$ algorithm that we did not implement. The most important of these improvements is the usage of language information. If this was implemented, the word suggestions could possibly improve a lot. The dynamic channel weighting by gesturing speed could also be useful if the keyboard was used a lot by some people. Because if they used it a lot, they might write more from memory recall than visually guided (\Cref{gesturing speed}), and it might make sense to weight one channel more than the other one.\\
In the future, one could also improve the available layouts. Right now, only the standard layout is modifiable, but no hexagonal layouts are possible. So, a future work could be to implement a possibilty to change from the standard layouts to hexagonal layouts.\\
Another thing for more user-friendliness was the use of error messages. For example, if a layout is faulty added to the file containing all layouts, the user could be informed about it in the VR space via some kind of text bubble.\\
Finally, another interesting implementation would be to be able to change the language. Right now, English is the only available language. There could be some new options button to let the user change the language.