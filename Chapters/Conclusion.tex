% !TEX root = ../Thesis.tex
\chapter{Conclusion}

In this last section, we will briefly talk about what we have done in this thesis, the results and what can be implemented in the future to improve the keyboard developed during this thesis.

\section{Conclusion}
We developed a word-gesture keyboard for VR. We used the algorithm of the $\text{SHARK}^2$ system \cite{Kristensson2004SHARK2AL} as a basis, which we implemented in a simplified version in Unity. Besides the basic function to write words with gestures, there are some other functionalities our word-gesture keyboard provides. A user can add new words, create new layouts, change the size of the keyboard and choose from up to four other words, if the best one is not the intended one. From the results, we think that the text input method we implemented might not be the fastest way to input words in VR. At least this can be said for beginners. When we tried to write some phrases, we got a fairly high WPM score, which shows that one could possibly write quickly after some training. Further, it seems to have a much lower error rate compared to single letter input methods. Either this is due to the way we instructed the participants at the evaluation, or it is because of how the keyboard works. All in all, we think it is a nice and useful text input method for the VR space.

\section{Future Work}
First of all, the word detecting algorithm of the keyboard can be extended with all the improvements used in the $\text{SHARK}^2$ algorithm that we did not implement. The most important of these improvements is the usage of language information. If this was implemented, the word suggestions could possibly improve a lot. The dynamic channel weighting by gesturing speed could also be useful if the keyboard was used a lot by some people. Because if they used it a lot, they might write more from memory recall than visually guided (\Cref{gesturing speed}), and it might make sense to weight one channel more than the other one.\\
In the future, one could also improve the available layouts. Right now, only the standard layout is modifiable, but no hexagonal layouts are implemented. So, a future work could be to implement hexagonal layouts and a possibilty to switch between the standard layouts and them.\\
Another thing for more user-friendliness would be the use of error messages. For example, if a layout is added incorrectly to the file containing all layouts, the user could be informed about it in the VR space via some kind of text bubble.\\
Finally, another important implementation would be to be able to change the language. Right now, English is the only available one. There could be some new options button to let the user change the language.